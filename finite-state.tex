\chapter{\label{ch:fsa}Finite state automata and regular languages}

The regular languages, the simplest class of languages
we defined in Chapter~\ref{ch:formal-languages},
and the corresponding abstract machines, finite state automata (FSA), 
have many practical uses ranging from circuit design to string search.
They are also an important tool for computational linguistics.
In this chapter we will study FSA and regular languages.

%TODO: applications - dictionaries, shallow parsing ...

%TODO: link to regular expressions

\section{Finite state automata}

A finite state automaton is an abstract machine with a finite number of states.
The machine can be in one of the states in a given time,
and it changes its sates as it processes an input string
from a finite alphabet.
If the machine is in one of final or accepting states at the end of the input,
it accepts the string.
As noted earlier, finite state automaton
recognize and generate regular languages.
A finite state automaton accepts all and the only strings
that are in the corresponding regular language.

\begin{marginfigure}
  \begin{center}
    \tikzset{external/export next=false}
    \begin{tikzpicture}[every initial by arrow/.style={thick},
      ]
      \node[draw,state,initial,initial text={}] at (0,0) (s0) {0};
      \node[draw,above right=of s0,state] (s1) {1};
      \node[draw,below right=of s0,state,accepting] (s2) {2};
      \draw[->,thick,>=stealth]
                (s0) edge[above,bend left] node (nl) {b} (s1)
                (s0) edge[below,bend right] node{a} (s2)
                (s1) edge[loop above] node{b} (q1);
      \draw[->,thick,>=stealth]
                (s1) edge[right,bend left=90] node{c} (s2);
      \node[blue] at (-1, -1) (ilab) {initial state};
      \draw[blue,->,shorten >=1pt]  (ilab) to[bend right] (s0);
      \node[blue] at (-1, 1) (tlab) {transition};
      \draw[blue,->,shorten >=1pt]  (tlab) to[bend left] (nl.south);
%      \draw[blue,->,shorten >=1pt]  (tlab) to[bend left] ($(s0)!0.5!(s1)$);
      \node[blue] at (0, 2) (slab) {state};
      \draw[blue,->,shorten >=1pt]  (slab.east) to[bend left] (s1.north west);
      \node[blue] at (0, -2) (flab) {accepting state};
      \draw[blue,->,shorten >=1pt]  (flab) to[bend right] (s2);
    \end{tikzpicture}
  \end{center}
    \caption{\label{fig:example-fsa}%
      An example finite-state machine.
    }
\end{marginfigure}
A common way to think about (and represent) an FSA is a directed graph,
where the nodes correspond to the states,
and edges are labeled with the symbols from the alphabet.
Figure~\ref{fig:example-fsa} presents an FSA represented as
a directed graph.
The machine starts at the start state (marked with an incoming arrow)
and moves between the states based on the given input.
If the machine is in a final state
(represented as a double circle)at the end of the input,
the input string is accepted.

\subsection{Deterministic finite automata}
Formally, a finite state automaton, $M$,
is a tuple $(\Sigma,Q, q_{0}, F, \Delta)$ with

\begin{itemize}[nosep]
  \item[$\Sigma$] is the alphabet, a finite set of symbols
  \item[$Q$] a finite set of states
  \item[$q_{0}$] is the start state, $q_{0} \in Q$
  \item[$F$] is the set of final states, $F \subseteq Q$
  \item[$\Delta$] is a function that takes a state and a symbol in the alphabet,
    and returns another state ($\Delta: Q\times\Sigma \ra Q$)
\end{itemize}

The important part of the this definition is the last item,
translates to the requirement that given a state and a symbol,
there is only one possible state to go.
As a result,
a deterministic automata requires
exactly one edge labeled by each symbol in the alphabet leaving every state.
