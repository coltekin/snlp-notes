\chapter{\label{chap:morphology}Computational morphology}

\begin{itemize}
  \item Motivation
    \begin{itemize}
      \item Lexicon size - data sparsity
      \item Predictions about unknown words
      \item Parsing / generation
      \item MT 
      \item Spell checking
      \item Information retrieval
      \item TTS
    \end{itemize}
  \item Some definitions
    \begin{itemize}
      \item root - stem - lemma - lexeme
      \item affix - suffix - prefix - infix - circumfix 
      \item reduplication
      \item zero - empty morphemes
      \item inflection vs.\ derivation\\
        tendencies:
        \begin{itemize}
          \item derivation changes category
          \item inflection tends to be paradigmatic
          \item inflection is productive
          \item derivation is lexical, inflection is functional
          \item inflection is semantically regular
          \item inflection is restricted syntactically - agreement etc.
          \item inflection tends to fall on periphery - derivation is more central
          \item derivation is often repeatable
        \end{itemize}
      \item morpheme / morph
      \item compounding 
      \item paradigm 
      \item free - bound morphemes, content vs.\ function
      \item regular and irregular morphology
    \end{itemize}
  \item Morphological typology
    \begin{itemize}
      \item Analytic - synthetic
      \item Isolating - inflectional - agglutinative - templatic
      \item General tendencies: like suffixing more than prefixing, \ldots
    \end{itemize}
  \item Morphological annotation
    \begin{itemize}
      \item Morphology in linguistic glosses
      \item Tagsets in treebanks
      \item Positional tags
      \item Morphological databases - CELEX
    \end{itemize}
  \item Stemming / lemmatization / segmentation
    \begin{itemize}
      \item Compounding / compound splitting
      \item Tries for finding affixes
    \end{itemize}
  \item Tasks in computational morphology
    \begin{itemize}
      \item recognition
      \item analysis
      \item generation
    \end{itemize}
  \item Finite state morphological processing
    \begin{itemize}
      \item FSA
      \item Example: English adjectives (?)
      \item FSA regex relation
      \item Regular languages and Chomsky hierarchy
      \item Deterministic/non-deterministic FSAs
      \item FST 
      \item FSTs as relations - properties
      \item Example FST
      \item Modeling orthographic/phonetic changes: context-sensitive rewrite rules
      \item Porter stemmer
      \item Two-level morphology
      \item Practical examples: Xerox, SFST, \ldots
      \item Weighted FSTs
      \item Weighted FSAs for spelling correction
    \end{itemize}
  \item Statistical learning of morphology
    \begin{itemize}
      \item Learning (re)inflection
      \item Unsupervised methods
    \end{itemize}

\end{itemize}
