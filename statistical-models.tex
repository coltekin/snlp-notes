\chapter{\label{chap:inference}Statistical models:
  learning, estimation, inference and prediction}

% If we know that we have a fair coin,
% we can calculate the probability of obtaining
% a certain number of heads in a sequence of coin tosses
% using probability theory.
% What if, we do not know if the coin is fair or not?
% Can we \emph{infer} whether the coin is fair or not
% from a finite number of outcomes?
% Similarly,
% if we know the election results in a country,
% probability theory can tells us
% the probability of a randomly picked the voters
% having voted a particular political party.
% But what can we say about the election results,
% given we know which parties a limited number of voters will vote?
% In general,
% given a finite sample of data,
% we want to know, or infer,
% some properties of the data source (e.g., population)
% that the data comes from.
% These questions fall into the area of \emph{statistical inference}.

Modeling natural phenomena is one of the basic activities in science.
We build models of a natural phenomenon to gain a better insight
into the phenomenon being modeled,
and predict aspects that we do not know about it.
Models are used in many fields in science and engineering.
Just to name a few well-known examples,
we note Galilean model of solar system in astronomy;
Bohr model of atom in physics;
animal models in medicine that are used for studying human diseases
on non-human animals;
econometric models in economics;
the formal models of atmospheric conditions used for weather forecasts; 
scaled physical models of bridges, cars, and other objects
that are used frequently in design and engineering of these objects.

Whether they are abstract (mathematical) models,
or physical ones,
these models allow us to study the phenomenon being modeled
in a convenient way.
They allow us to test hypotheses and make \emph{inferences}
or \emph{predictions} that we cannot do directly
on the real object being studied.

In this lecture, we are interested in
\emph{statistical model}s
which constitute a very broad range of models used in many disciplines.
In a statistical model,
part of the phenomenon being modeled is subject to uncertainties.
As a result,
the data we collect or observe does not lead to fully deterministic model. 
Statistical models are the main subject of study
in statistics and machine learning,
and naturally, has a very important role in natural language processing.

Before this general introduction to statistical models,
we remind that ``all models are wrong, some are useful.''%
\autocite[p.~424]{box1986}
In any modeling effort,
one has to make simplifications or assumptions that do not match
with the object or the process being modeled.
However,
as long as we are aware of its shortcomings,
we can build useful models.

\section*{Where to go from here}

\textcite{wasserman2004}
