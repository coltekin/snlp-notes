\chapter{\label{ch:formal-languages}Formal languages and automata}

Natural languages can hardly be considered `formal' languages.%
\sidenote{?? Quote from Montague: ``no important theoretical difference between natural languages and the artificial languages of logicians.''}
From a practical engineering perspective,
our efforts to come up with (hand-crafted) formal specifications of
languages often often resulted in sub-optimal results.
However, we learned a lot about trying to do so,
and many of the concepts/methods developed are at the hart of computer science,
and many (practical) methods we use in computational linguistics.

In this lecture,
we will go through a gentle introduction to the theory of computation.
Our focus will be on \emph{automata} and corresponding formal languages.
However, we will necessarily touch on the other two subfields of
theory of computation, \emph{computability} and \emph{complexity}.

\section{Some definitions: Languages, strings, grammars}

\newthought{An alphabet} is a set of symbols.
We often use letters for our examples,
so $\{a, b, c\}$ is an alphabet consisting of three letters.
For many problems we may be interested in a binary alphabet like $\{1, 0\}$.
And for some of natural language problems our alphabet may consist of words,
such as $\{\text{cat}, \text{dog}, \text{sat}, \text{on}, \text{mat}\}$.
By convention, the Greek capital letter sigma $\Sigma$ for representing 
an alphabet.

\newthought{A string} is a finite sequence of symbols from an alphabet.
$a$, $ac$ or $abcaab$
are example string over alphabet $\{a, b, c\}$.
A special string that does not contain any symbols,
the \emph{empty string} is denoted by Greek letter epsilon, $\epsilon$.
We denote the set of all string over $\Sigma$,
including the empty string, with $\Sigma^{*}$.
$\Sigma^{+}$ is often used for the set of non-empty string in $\Sigma^{*}$
(a shorthand for $\Sigma^{*} - {\epsilon}$).
Note that both $\Sigma^{*}$ and $\Sigma^{+}$ have infinite number of members.
We indicate the length of a string with vertical bars on both sides,
$\lvert{}ab\rvert = 2$, or if $u = 0100$, $\lvert{}u\rvert{} = 4$.
For two string $u$ and $v$, we indicate their concatenation with $uv$.
If $u = 0100$ and $v = 11$, $uv = 010011$.

\newthought{A language} is a set of strings over an alphabet.
Empty set, $\emptyset$, is a language,
as well as the set containing only the empty string $\{\epsilon\}$.
Similarly $\Sigma^{*}$ is another language over alphabet $\Sigma$.
We are generally interested in more interesting languages,
such as%
\sidenote{Although we define these languages informally here,
  all of them can be defined more formally.
  For example, the language by the first item
  can be defined as $\{u\mid \lvert{}u\rvert = 2\}$.}
\begin{itemize}
  \item The set of strings of length \num{2}
    over $\{0, 1\}$: $\{00, 01, 10, 11\}$
  \item The set of strings with even number of \num{1}'s over $\{0, 1\}$:
      $\{\epsilon, 101, 0, 11, 111110, \ldots\}$
  \item The set of string that retain alphabetical ordering
    over $\{a, b, c\}$: $\{a, ab, abc, ac, abcc, \ldots\}$
  \item The set of strings of words
    that form a grammatically correct English sentences
\end{itemize}
The strings that are in a language are called \emph{sentences} of the language.

\newthought{A grammar} is a finite description of a language.
Although the set-based definitions above is flexible,
it is more convenient to express a language with a finite set of rules.
The most common method for formalizing grammars is based on
a set of \emph{rewrite rules}.
A rewrite rule two sides delimited by an arrow ($\ra$).
The grammar generates strings in the language,
by rewriting sequence of symbols matching right-hand-side (RHS) of the rule 
as the left-hand-side (LHS) of the rule,
starting from a special `start' symbol denoted by `S'.
The sequences of symbols in a rewrite rule is may consist of
\emph{terminal} symbols that are member of the alphabet
and \emph{non-terminal} symbols.
The non-terminal symbols are denoted using capital letter by convention.
For example, the grammar in Figure~\ref{fig:exmpl-grammar}
produces a language over the alphabet $\{a, b, c\}$
whose last symbol (word) is always an $a$.
This is not the only way to express the same language.
There are multiple (better, more concise) ways of writing
the same grammar.
The rule number 2\ in the grammar shown in Figure~\ref{fig:exmpl-grammar}
is recursive, it allows infinite number applications of this rule,
and hence, the grammar defines an infinite language with a finite set of rules.

\begin{marginfigure}
  \begin{tcolorbox}
    \begin{enumerate}
      \item $ S \ra A a$
      \item $ A \ra B A$
      \item $ A \ra B$
      \item $ B \ra a$
      \item $ B \ra b$
      \item $ B \ra c$
    \end{enumerate}
  \end{tcolorbox}
  \caption{\label{fig:exmpl-grammar}%
    An example grammar producing string over $\{a, b, c\}$
    that end with symbol $a$.
    A common short-hand for writing last three rules is a single rule 
    of the form `$B \ra a | b | c$'.
  }
\end{marginfigure}

Formally,
a phrase structure grammar is a tuple
$G = (\Sigma, N, S, R)$, where
  \begin{itemize}
    \item[$\Sigma$] is an alphabet of terminal symbols
    \item[$N$] are a set of non-terminal symbols 
    \item[$S$] is a special `start' symbol $\in N$
    \item[$R$] is a set of rules of the form $\alpha \ra \beta$,
      where  $\alpha$ and $\beta$ are strings from $(\Sigma \cup N)^{*}$.
  \end{itemize}

For our example in Figure~\ref{fig:exmpl-grammar},
$\Sigma = \{a, b, c\}$, $N = \{S, A, B\}$, $S = S$,
and $R = \{ S \ra A a, A \ra B A, A \ra B,
      B \ra a, B \ra b, B \ra c\}$.


\begin{marginfigure}
  \begin{tcolorbox}[halign=center]
    \begin{tabular}{l@{\Ra}ll}
      $S      $ & $A a$     &(rule 1)\\
      $A a    $ & $B A a$   &(rule 2)\\
      $B A a  $ & $B B A a$ &(rule 3)\\
      $B B A a$ & $B B B a$ &(rule 3)\\
      $B B B a$ & $B B c a$ &(rule 6)\\
      $B B c a$ & $B b c a$ &(rule 5)\\
      $B b c a$ & $a b c a$ &(rule 4)\\
    \end{tabular}
  \end{tcolorbox}
  \caption{\label{fig:example-derivation}%
    A derivation of string $abca$ using the grammar
    in Figure~\ref{fig:exmpl-grammar}.
  }
\end{marginfigure}
Given the grammar in Figure~\ref{fig:exmpl-grammar},
the string \texttt{abca} can be derived with the derivation sequence
presented in Figure~\ref{fig:example-derivation}.
We could write the same derivation sequence as
'$S \Ra A a \Ra B A a \Ra B B A a \Ra B B B a \Ra B B c a \Ra B b c a \Ra a b c a$'.
In general, if a string can be derived from the start symbol S
using a sequence of rewrite rules as above
(sometimes notes as `S $\overset{*}{\Ra}$ a b c a',
we say that the grammar \emph{generates} the string,
or the string is \emph{recognized} by the grammar.
Hence, the string is a valid sentence in the language.

The above definitions are important for understanding formal languages,
and also used in study of natural languages in formal linguistics.

\section{Chomsky hierarchy of languages}

\begin{marginfigure}
  \begin{center}
    \tikzset{external/export next=false}%
    \begin{tikzpicture}[x=4mm,y=8mm]
      \draw (0,0) ellipse (3.0 and 1.0);
      \node[font=\footnotesize] at (0,0) {Regular};
      \draw (0,0.5) ellipse (4.0 and 1.5);
      \node[font=\footnotesize] at (0,1.3) {Context Free};
      \draw (0,1.0) ellipse (5.0 and 2.0);
      \node[font=\footnotesize] at (0,2.3) {Context Sensitive};
      \draw (0,1.5) ellipse (6.0 and 2.5);
      \node[font=\footnotesize] at (0,3.2) {Recursively Enumerable};
      \draw (-6.1,-1.1) rectangle (6.1,4.1);
      %\draw (4,3.8) node {All};
    \end{tikzpicture}
  \end{center}
  \caption{\label{fig:chomsky-h}%
    Chomsky hierarchy of language classes. 
    The classes are also known as \emph{type-0}
    (recursively enumerable) to \emph{type-3} (regular).
  }
\end{marginfigure}

The Chomsky hierarchy of languages
is one of Chomsky's most important contributions
to both linguistics and computer science \parencite{chomsky1959b}.
For his theory of natural language syntax,
Chomsky defined a set of formal language classes
(depicted in Figure~\ref{fig:chomsky-h}).
Each class in this hierarchy is the proper subset of the larger class.
The larger classes are more descriptive,
but their computational processing is more demanding.
This particular classification of the language classes
has some attractive formal properties,
and each class corresponds to
a certain type of abstract computational device
that is capable of recognizing
and generating the languages in the corresponding class.
As a result, as well as its importance for (computational) linguistics,
this hierarchy is also an important part theory of computation.
We will define the individual language classes and the
properties of the rewrite grammars that expresses them.
In general, we will describe each language class
based on the restriction they put in the rewrite grammars that describe them.

We will next discuss at these language classes with in detail.
However, we note that these language classes are not 
the only formally defined language classes.
There are other well-studied language classes
some of which can be placed
in the set-inclusion hierarchy in Figure~\ref{fig:chomsky-h},
while other may cross-cut parts of this hierarchy.

\subsection{A brief divergence: complexity theory}

As noted earlier,
study of formal languages strongly linked to theory of computation,
particularly complexity theory.
Here, we will briefly review a part of the complexity theory briefly,
since some of the concepts will be instrumental in our discussion
of the language classes and the computational problems
related to these language classes.

In analysis of algorithms,
and algorithms worst time order of complexity 
is indicated a notation called \emph{big-O notation}.
Big-O notation aims to measure an algorithms complexity
as its input grows in size.
Since the aim of the big-O notation is indicating
the worst case with large number of input,
we discard all constant and lower order complexities from
the analysis of our algorithm.
For example, if our analysis indicate an algorithm to require
$T(n) = 100\times{}2^{n} + n^{10} + \log{n} + 10^{100}$ steps
(e.g., a simple operation on a CPU) to complete,
the complexity we are interested in is indicated by $O(2^{n})$.
In general, we drop all terms except $2^{n}$ above,
since all other figure will dwarfed by this term
as the size of the input ($n$) is increased.
Although even a constant factor is often important in practice,
we may be able get reasonable running times by parallelization,
or hope that increase in computing power will allow us
to use these algorithms.
The order of complexity measured with big-O notation has
more important theoretical consequences.
A few common examples of complexities measured using big-O notation
are given below.

\begin{itemize}[nosep,labelindent=1em,labelwidth=1.3cm,labelsep*=1em,leftmargin =!]
  \item[\textcolor{blue!50!green}{$O(1)$}] constant,
    an algorithm that does not depend on the size of its input,
    e.g., accessing a single element of an array
  \item[\textcolor{blue!50!green}{$O(\log n)$}] logarithmic,
    e.g., binary search
  \item[\textcolor{blue!50!green}{$O(n)$}] linear,
    e.g., finding minimum or maximum value of an array
  \item[\textcolor{blue}{$O(n \log n)$}] log linear,
    e.g., some sorting algorithms 
  \item[\textcolor{blue}{$O(n^{2})$}] quadratic,
    e.g., selection sort (and most arithmetic operations)
  \item[\textcolor{blue}{$O(n^{3})$}] cubic
  \item[\textcolor{red}{$O(2^{n})$}] exponential,
    e.g., graph search algorithms involving backtracking
    (note that the base is not important)
  \item[\textcolor{red}{$O(n!)$}] factorial,
    e.g., enumerating all binary trees spanning $n$ items (words)
\end{itemize}

\begin{marginfigure}
  \begin{tikzpicture}
    \begin{axis}[
        ymax=2000,
        axis x line=bottom,
        axis y line=left,
        xlabel=$n$,
        xlabel style={font=\tiny},
        ylabel style={font=\tiny},
        title style={font=\tiny},
        ticklabel style={font=\tiny},
        x=0.4mm,
        y=0.025mm,
%        cycle list={colormap/Spectral-7},
        thick,
        grid,
        legend style={font=\tiny,fill=none,draw=none},
        legend cell align=left,
      ]
      \addplot[blue,dashed, domain=1:100, samples=100] {ln(x)};
      \addplot[green,domain=1:100, samples=100] {x};
      \addplot[black,dashed, domain=1:100, samples=100] {x*ln(x)};
      \addplot[orange,domain=1:100, samples=100] {x^2};
      \addplot[brown,dashed,domain=1:15, samples=100, smooth,unbounded coords=jump] {x^3};
      \addplot[purple,domain=1:12, samples=100, smooth,unbounded coords=jump] {2^x};
      \addplot[red,dashed,domain=1:8, samples=7, unbounded coords=jump] {x!};
      \legend{
          $O(\log n)$,
          $O(n)$,
          $O(n \log n)$,
          $O(n^{2})$,
          $O(n^{3})$,
          $O(2^{n})$,
          $O(n!)$,
      };
    \end{axis}
  \end{tikzpicture}
  \begin{tikzpicture}
    \begin{semilogyaxis}[
        ymax=1000000000000000000000000000,
        ytickten={0,5,...,25},
        axis x line=bottom,
        axis y line=left,
        xlabel=$n$,
        xlabel style={font=\tiny},
        ylabel style={font=\tiny},
        title style={font=\tiny},
        ticklabel style={font=\tiny},
        x=0.4mm,
        y=0.75mm,
%        cycle list={colormap/Spectral-7},
        thick,
        grid,
        legend style={font=\tiny,fill=none,draw=none, yshift=-1cm},
        legend cell align=left,
      ]
      \addplot[blue,dashed, domain=1:100, samples=100] {ln(x)};
      \addplot[green,domain=1:100, samples=100] {x};
      \addplot[black,dashed, domain=1:100, samples=100] {x*ln(x)};
      \addplot[orange,domain=1:100, samples=100] {x^2};
      \addplot[brown,dashed,domain=1:100, samples=100, smooth,unbounded coords=jump] {x^3};
      \addplot[purple,domain=1:100, samples=100, smooth,unbounded coords=jump] {2^x};
      \addplot[red,dashed,domain=1:30, samples=30, unbounded coords=jump] {x!};
      \legend{
          $O(\log n)$,
          $O(n)$,
          $O(n \log n)$,
          $O(n^{2})$,
          $O(n^{3})$,
          $O(2^{n})$,
          $O(n!)$,
      };
    \end{semilogyaxis}
  \end{tikzpicture}
  %TODO: more print-friendly graphs
  \caption{\label{fig:computational-complexity}%
    Order of operations required by different algorithmic complexity
    measured by big-O notation.
    Top graph represents this linearly,
    while the y-axis of bottom graph is logarithmic
    (changes in the graph indicate an exponential changes
    in the number of operations),
    allowing us to visualize later compare complexity over
    a larger interval of input size $n$.
  }
\end{marginfigure}
Naturally, we prefer the lower-complexity algorithms
(earlier items in the above list).
However, the behavior of the last two items in the list
and the earlier ones are drastically different when the input size is increased.
The ones up to $O(n^{3})$ above,
in general any algorithm whose complexity can be explained
a polynomial of any degree,
are called \emph{polynomial} time algorithms.
In general, polynomial time algorithms are said to be \emph{tractable}.
And algorithms with worse complexity than polynomial time
are called \emph{intractable}.
Intractable algorithms scale badly with increased input size,
Even with reasonable increase in computing power,
we do not have any hopes of solving intractable algorithms
for larger input sizes.
Figure~\ref{fig:computational-complexity}
presents order of operations needed for algorithms
with different complexities.
The top graph in Figure~\ref{fig:computational-complexity},
showing how drastic the complexity may change.
Note that an algorithm with $O(n^{3})$ requires more operations
for small input sizes,
but in time non-polynomial algorithms more costly for larger input sizes.
The bottom graph shows this more drastically,
as the input size increased,
the polynomial and non-polynomial algorithms becomes much clear.

A fundamental activity in computer science is to find
algorithms with lower big-O complexity.
For example,
a brute-force approach to recognizing context-free languages 
is exponential in the length of the sentence (or computer program).
However, we have clever solutions that work in polynomial time
using a modest amount of additional storage.
The class of problems that has a known polynomial-time algorithm is
indicated by P.
For some problems we do not know a tractable algorithm.
However, they are not provably intractable,
we may hope to find some polynomial time algorithm for solving
these problems and they may become a member of P.
An interesting subclass of these problems
for which we do not have a polynomial-time algorithm to solve it,
we can identify the solution (in polynomial time) when we see the solution.
This class of problems are called
\emph{non-deterministic polynomial time} (NP).
P is a subset of NP.
However, one of the big questions in theory of computation is
whether P is equal to NP or not.
If it is,
then there are tractable algorithms waiting to be discovered
for many of the problems in NP.

An interesting sub-class of NP is called NP-complete.
If we find a polynomial time solution to any of the NP-complete problems,
we can solve all NP problems in polynomial time,
since solution of any NP problem can be `converted' obtained using
the solution of an NP-complete problem
with a polynomial overhead for the conversion.
Finding a polynomial algorithm for an NP-complete problem
would prove P = NP,
as well as solving many difficult problems in a wide range of fields.
However, 
the fact we do not know tractable algorithms for NP-complete problems
despite decades of efforts by many clever people
suggests that P is probably not equal to NP.
Hence, when we face a problem that is NP-complete,
or in a more difficult complexity class,
chances of finding a scalable solution is rather unlikely.

The concepts introduced above should be sufficient to understand
some of the complexity issues we will discuss below.
However, there are many other complexity classes,
and many unknowns, interesting problems,
about the relation between these classes.
A reasonable overview of computational complexity is
out of the scope of this lecture.
The additional resources at end of the chapter points to
a few textbooks covering the complexity theory is covered.

\subsection{Regular languages}

Regular languages is the simplest
(least expressive, easy to process computationally)
class in Chomsky hierarchy.
The regular languages are defined by \emph{regular grammars}
which are (mostly) equivalent to \emph{regular expressions}
that are used in many programming environments
for searching patterns in text.
We will study regular languages in detail in Chapter~\ref{ch:fst}.

\begin{marginfigure}
  \begin{tcolorbox}[rounded corners]
    \begin{multicols}{2}
      \begin{enumerate}
        \item $A \ra a$
        \item $A \ra Ba$
        \item $A \ra \epsilon$
      \end{enumerate}
      \begin{enumerate}
        \item $A \ra a$
        \item $A \ra aB$
        \item $A \ra \epsilon$
      \end{enumerate}
    \end{multicols}
  \end{tcolorbox}
  \caption{\label{fig:regular-grammars}%
    The possible rules for left-regular (left)
    and right-regular (right) grammars.
    For both, $A$ and $B$ are non-terminal,
    $a$ is a terminal sybol ($a \in \Sigma$),
    and $\epsilon$ is the empty string.
  }
\end{marginfigure}
A \emph{regular grammar} is a grammar that generates a regular language.
There are two flavors of regular grammars.
The \emph{left-regular grammars} can have only the rules listed in left side
of Figure~\ref{fig:regular-grammars}.
The rules allowed in a \emph{right-regular grammar} is listed
in the right side of Figure~\ref{fig:regular-grammars}.
In simple words, regular grammar can have three types of rules
\begin{itemize}
  \item A single non-terminal symbol rewritten as a single terminal symbol
  \item A single non-terminal symbol as one terminal
    and one non-terminal symbol
    (the order determines whether the grammar left or right regular)
  \item A single non-terminal symbol as empty string ($\epsilon$)
\end{itemize}
Note that both definitions allow defining all (and only) regular languages.
Some definitions do not allow transition to empty string
(3rd rule in both definitions).

A regular grammar only defines regular languages.
However, note that regular languages can also be defined
with non-regular grammars (e.g., context free grammars we will define next).

%TODO: examples, exercises

\subsection{Context-free languages}

Context-free languages are the next (more expressive) language
after regular languages in the Chomsky hierarchy.
They are used in describing (and parsing) programming languages.
The syntax of all programming languages are based on context-free grammars.%
\sidenote{They are generally defined by a restricted
  (deterministic, non-ambiguous) subset of context-free grammars
  for efficiency of parsing,
  and in a limited/restrict set of non-context freeness may be used. 
}
Similarly,
many syntactic theories of natural languages are based on
a context-free backbone
(we will discuss this further in Section~\ref{sec:nl-and-fl}).

The restrictions on the rewrite rules
for context-free grammars are simple:
left-hand-side (LHS) of the rule has to contain a single non-terminal.
Unlike regular grammars,
we do not have any restrictions on right-hand-side (RHS)
of the rewrite rules.
They can contain any combination of non-terminal and terminal symbols
(or $\epsilon$).
The context-free languages are called so,
since the rewrite rules express
expanding a non-terminal anywhere independent of their context.

%TODO: examples, exercises

\subsection{Context sensitive languages}

Context sensitive languages are a superset of context-free languages.
The rules of context sensitive language are required to be in the following form.
\[
  \alpha A \beta \ra \alpha \gamma \beta
\]
where $A$ is a single non-terminal symbol,
$\alpha$ and $\beta$ are
possibly empty strings of terminals and non-terminals,
and $\gamma$ is a non-empty string of terminal and non-terminal symbols.
Note that the above definition does not allow
a LHS non-terminal to be deleted (rewritten as $\epsilon$).
However, a rule of the form $S \ra \epsilon$ is allowed
to ensure that the context-sensitive languages are
a superset of context-free languages.
Another way to define a context sensitive grammars is 
requiring the RHS of all rules to be longer than or equal to the LHS.%
\sidenote{The grammars defined this way is known as
  \emph{non-contracting grammars} and are
  known to be \emph{weakly equivalent} to context-sensitive grammars.
  That is, we can convert any non-contracting grammar to 
  a context-sensitive grammar.
}

Many problems with context-sensitive grammars do not have
efficient solutions (see Section~\ref{sec:formallang-automata}).

\subsection{Recursively enumerable languages}

The recursively enumerable languages are the most expressive language
in the Chomsky hierarchy.
Recursively enumerable languages are languages over an alphabet that are,
well, recursively enumerable.
That is, there exist an algorithm that can enumerate (list)
all strings in the language.

The grammars that define recursively enumerable languages
do not pose any restrictions on their rewrite rules,
except that their LHS being non-empty.
The class of grammars that generate recursively enumerable languages
are called \emph{unrestricted grammars}.

\section{\label{sec:formallang-automata}Formal languages and automata}

The simplest question we ask about a grammar,
or the corresponding language,
is recognizing strings that belong to the language.
That is, given a string,
can we determine whether it belongs to the language or not?

This question links the language classes described above to
abstract computation devices, \emph{automata}.
In other words,
for each language class discussed above,
there is an type of automaton that recognizes the strings of
a language in the class.
We will briefly discuss these correspondences,
starting from the most-expressive language class.

Strings in a recursively enumerable language,
the most expressive class in the Chomsky hierarchy,
can be recognized by a \emph{Turing machine}.
The Turing machine is a simple abstract computing device
that can compute any computable function.
In other words, any algorithm can be simulated by a Turing machine.
The relationship between recursively enumerable languages
and Turing machines mean
that a Turing machine can generate all (and only) string
in the corresponding recursively enumerable language.
However, the question is whether a string is in
a recursively enumerable language or not is undecidable,
since it is equal to a \emph{halting problem},
a well known undecidable problem.

The machines that recognize the context-sensitive languages
are a restricted version of Turing machines
called \emph{linear-bounded automata}.
Although whether a string is generated by a context-sensitive grammar
or not is decidable,
it is known to be computationally complex.%
\footnote{More formally,
  it is PSPACE-complete, and PSPACE is a superset of NP.
}

Context-free grammars can be recognized by \emph{push-down automata}.
A push-down automaton has a finite number of sates that controls
the computation
(similar to the finite-state machines we will summarize next),
and a stack (with access to only the top element) as memory.
Although the complexity of first two
the grammars and machines discussed above are computationally intractable,
there are many practical algorithms for solving problems
involving context-free grammars.
We will revisit the context-free grammars
when we discuss parsing algorithms later.

\begin{marginfigure}
  \begin{center}
    \begin{tikzpicture}[
      ]
      \node[draw,thick,state,initial,initial text={}] (s0) {0};
      \node[draw,thick,right=of s0,state] (s1) {1};
      \node[draw,thick,right=of s1,state,accepting] (s2) {2};
      \draw[->] (s0) edge[above] node{a} (s1)
                (s1) edge[loop above] node{b} (q1)
                (s1) edge[above] node{c} (s2);
    \end{tikzpicture}
  \end{center}
  \caption{\label{fig:fsa-example}%
    An example automata recognizing the strings over alphabet
    $\{a, b, c\}$ which begins with $a$, followed by zero or more
    $b$s, and ends with $c$.
  }
\end{marginfigure}
Finally, the regular languages are recognized by finite-state automata.
Finite state automata (FSA) can be described by a directed graph
with a finite nodes (states) and edges are
labeled with symbols of the alphabet.
A finite-state automaton
can accept (or rejects) its input in time complexity
linear in the length of input.
Furthermore, the FSA are `memoryless',
except the automaton itself, they do not need any additional memory.
Figure~\ref{fig:fsa-example} shows an example finite state automata.
We will discuss finite-state automata in Chapter~\ref{ch:fsa}.

\section{Where do natural languages fit}

One of the motivations behind building the classes of languages
in the Chomsky hierarchy was to characterize
the type of formal grammars that are adequate for natural languages,
primarily natural language syntax.
By `adequate', we mean classes of grammars that can represent
the phenomena we observe in natural languages,
yet computationally not so demanding.
The latter condition, computational efficiency,
is not only desirable
from the perspective of computational processing of natural languages.
It is also important for cognitive science
where one of the main tenets is that cognition is a form of computation.
Since human cognitive system is a computing device
with limited/finite resources,
there are bounds on its computational capabilities.
As a result, the computational efficiency of language processing is
also important for study of human languages from a cognitive perspective.


Initially,
context-free languages were considered to be adequate
for representing natural languages.
However,
a small number of linguistic constructions cannot be represented
using context-free languages.
A typical example is the cross-serial dependencies found
in Dutch and Swiss German.
Figure~\ref{fig:cross-serial} presents
an example cross-serial dependency from Swiss German.
The dependencies with the verbs and their arguments cross.
As a result, context free grammars cannot represent this structure.
The problem is clearer if you consider
the operations of a push down automaton.
If we push the case marking on each noun on the stack,
by the time we process the verb \xmpl{hälfed},
we do not have access to case marking of its argument
as it is not on the top of the stack anymore.

\begin{figure*}
  \tikzset{external/export next=false}
  \begin{center}
    \begin{tikzpicture}
      \matrix[matrix of nodes] (m) {
        Jan & säit  & das & mer &\color{blue} em Hans & \color{red}es huss & \color{blue} hälfed & \color{red}aastriiche \\
        Jan & said & that & we & Hans (\textsc{dat}) & the house (\textsc{acc}) & helped & paint \\
      };
      \draw[blue] (m-1-5) to[in=north, out=north, looseness=0.5] (m-1-7);
      \draw[red] (m-1-6) to[in=north, out=north, looseness=0.5] (m-1-8);
    \end{tikzpicture}
  \end{center}
  \caption{
    An example cross-serial dependency from Swiss German
    \parencite{shieber1985}.
    The case marking of \xmplt{em Hans}{Hans} agrees with the verb
    \xmplt{hälfed}{helped},
    while the case marking of \xmplt{es huss}{house} agrees with the verb
    \xmplt{aastriiche}{paint}.
    Note that this resembles 
    $a^{\color{blue}n}b^{\color{red}m}c^{\color{blue}n}d^{\color{red}m}$.
  }\label{fig:cross-serial}%
\end{figure*}

Although the syntactic phenomena such as the one presented
in Figure~\ref{fig:cross-serial} are clearly non-context free,
their existence do not require the full representation power
(and computational complexity) of context-sensitive grammars.
A common theoretical consensus seems that
a grammar class, called \emph{mildly context sensitive} grammars
are adequate for representing natural languages
(see Figure~\ref{fig:chomsky-h-mcs}).
In practice,
many computationally oriented theories of grammars use
a context-free backbone with some extensions that increase their
expressive power.
This extension may range from mildly context-sensitive grammars%
\sidenote{E.g., Combinatory categorial grammar (CCG)
and tree adjoining grammar (TAG).}
to Turing machine equivalence.%
\sidenote{E.g., Head-driven phrase structure grammars (HPSG),
  and (according to some analyses lexical-functional grammar (LFG).
}
Although these extensions increase
their worst-case computational complexity drastically,
these formalisms often work acceptably in practice due to their
reasonable expected/average time complexity.


Before closing this brief discussion of natural languages
and formal languages,
their representational power and computational complexity,
we should note that the non-context-free extensions
to the theories of natural language grammars are not
motivated only by the representation power.
Besides being able to represent the constructions
like the cross-serial dependencies,
extensions requiring higher
than context-free power are often proposed
when they match better with linguistic intuitions.
In other words, even if a context-free grammar may be adequate
to represent a language,
an extended version may match the linguists' intuition
of how the linguistic structures are derived better.

\begin{marginfigure}
  \begin{center}
    \tikzset{external/export next=false}%
    \begin{tikzpicture}[x=4mm,y=8mm]
      \draw[dashed] (0,.6) ellipse (4.2 and 1.6);
      \draw (0,0) ellipse (3.0 and 1.0);
      \node[font=\footnotesize] at (0,0) {Regular};
      \draw (0,0.5) ellipse (4.0 and 1.5);
      \node[font=\footnotesize] at (0,1.3) {Context Free};
      \draw (0,1.0) ellipse (5.0 and 2.0);
      \node[font=\footnotesize] at (0,2.4) {Context Sensitive};
      \draw (0,1.5) ellipse (6.0 and 2.5);
      \node[font=\footnotesize] at (0,3.2) {Recursively Enumerable};
      \draw (-6.1,-1.1) rectangle (6.1,4.1);
      %\draw (4,3.8) node {All};
    \end{tikzpicture}
  \end{center}
  \caption{\label{fig:chomsky-h-mcs}%
    A depiction of Chomsky hierarchy of language classes
    (Figure~\ref{fig:chomsky-h},
    with the addition of \emph{mildly context-sensitive languages}
    represented with the dashed ellipse.
  }
\end{marginfigure}

\section{\label{sec:formal-lang-learnability}%
        Formal languages and learnability}

A central question in modern linguistics, surrounded with a big debate,
has been learnability of natural languages.
Some researchers, primarily based on Noam Chomsky's ideas,
championed the idea that the natural languages are not learnable.
Hence, only way for a human child to learn the languages they are exposed to
is to have an innate linguistic component.
The debate is linked to an even bigger, and much older one:
nature vs.\ nurture debate. 
The debate at large  has little practical use in a course
on natural language processing.
It also seems the heated discussion on the topic has slowed down after 2000s.
However, part of the debate is linked with some of the results regarding
learnability of classes of formal languages,
and together with the fact that this discussion has
a historical significance in the field,
it deserves a brief mention here.

There has been a number of formal proofs
indicating that none of the languages in the Chomsky hierarchy is
learnable only from positive examples.%
\footnote{Mainly due to \textcite{gold1967}.}
These results frequently used as an argument for the innateness hypothesis.
However, the formal arguments are often too general,
and do not necessarily match the settings in which children learn languages.
For example, the results cited above are obtained for the language classes
in the Chomsky hierarchy,
while the human languages do not necessarily match with
one of these language classes.
It is more likely that the class of natural languages,
if such a class exists,
cross cut the hierarchy (as in Figure~\ref{fig:chomsky-h-cross}).
In other words,
even regular languages allow some constructions
that are not useful for representing structures in human languages.
%TODO: example
Furthermore,
the condition that the languages should be learned from
the positive examples alone is likely to be too strict
for a realistic setting for child language acquisition.

\begin{marginfigure}
  \begin{center}
    \tikzset{external/export next=false}%
    \begin{tikzpicture}[x=4mm,y=8mm]
      \definecolor{vlgray}{rgb}{0.9,0.9,0.9}
      \filldraw[vlgray,draw opacity=0.1] (0,.6) ellipse (1.5 and 1.6);
      \draw[dashed] (0,.6) ellipse (4.2 and 1.6);
      \draw (0,0) ellipse (3.0 and 1.0);
      \node[font=\footnotesize] at (0,0) {Regular};
      \draw (0,0.5) ellipse (4.0 and 1.5);
      \node[font=\footnotesize] at (0,1.3) {Context Free};
      \draw (0,1.0) ellipse (5.0 and 2.0);
      \node[font=\footnotesize] at (0,2.4) {Context Sensitive};
      \draw (0,1.5) ellipse (6.0 and 2.5);
      \node[font=\footnotesize] at (0,3.2) {Recursively Enumerable};
      \draw (-6.1,-1.1) rectangle (6.1,4.1);
      %\draw (4,3.8) node {All};
    \end{tikzpicture}
  \end{center}
  \caption{\label{fig:chomsky-h-cross}%
    A possible set (indicated by the shaded region in the figure)
    for class of natural languages in the Chomsky hierarchy,
    to demonstrate that learnability results on a particular language
    class in the hierarchy is not necessarily
    a positive or negative indication for learnability of natural languages. 
    As in Figure~\ref{fig:chomsky-h-mcs},
    the dashed ellipse represents
    the class of mildly context-sensitive languages.
  }
\end{marginfigure}

Perhaps more interesting for computational linguistics is
the question of learnability of languages is an empirical question
that can be answered by a computational model.
Theoretical claims aside,
if we can build models that can learn natural languages
from the input available to children,
that is a good indication that they are learnable.
Even if the computational model may not mimic
the way children learn languages,
they can be used to show that learning is in principle possible.
Furthermore,
if we constrain such models the way human learners are constrained,
we may be able to answer even more interesting questions.

In summary,
although it is not part of the mainstream
computational linguistics or natural language processing fields,
the methods used typically in these fields can also be instrumental
in study of languages from cognitive science perspective.

\section{Where to go from here}

This lecture is a brief overview of formal languages
and automata theory.
We also noted some topics in theory of computation
(e.g., computational complexity and complexity classes),
and even some related issues in theoretical linguistics
and cognitive science.
All of these topics are, necessarily, covered briefly, on the surface.
Some of these topics will be revisited in later lectures.
Here, we point to some of the relevant literature.

\textcite{hopcroft1979} (and its successive editions) has been
the most influential textbook on formal languages and automata theory.
\textcite{sipser2006} is another good textbook
on the topic
covering formal languages and automata as well as 
complexity theory and other interesting issues in the theory of computation.

The discussion of learnability
briefly touched in Section~\ref{sec:formal-lang-learnability}
started in 1950s and 1960s based on Chomsky's work.
The literature in the field is vast,
ranges from psycholinguistics to computational linguistics.
Probably most popular work
on the innateness side of the debate is \textcite{pinker1994},
focusing mainly claims based on theoretical and experimental work.
\textcite{a.clark2011book} provides a more recent review of the field
from the opposite side of the debate.
This book also is more closely related
to the formal languages discussed in this chapter.

\section{What to add}

\begin{itemize}
  \item About complexity: infinity is overrated
  \item A bit of clarification/emphasis on `generative' side of grammars
\end{itemize}
