\chapter{\label{chap:parsing}Parsing}


In the next lecture, we will discuss \emph{parsing}, 
automatically annotating a given natural language sentence
with a syntactic representation.
Here, we briefly give a general introduction to parsing,
and introduce some terminology and applications.
After which, we will turn our attention
to two major syntactic representations that are used
in the computational linguistics,
\emph{constituency} and \emph{dependency} trees,
and how to parse sentences to produce these representations.%
\footnote{%
  We use these two terms rather in a theory-neutral way
  (as much as possible),
  and focus on the computational processes,
  and practical applications rather than their linguistic adequacy.
  Many linguistic theories of syntax
  make use of ideas from both approaches.
}

\section*{Where to go from here}

\begin{itemize}
  \item \textcite{lease2006} - for applications of parsing
  \item \textcite{muller2016} - for a introduction to different grammar formalisms
\end{itemize}
